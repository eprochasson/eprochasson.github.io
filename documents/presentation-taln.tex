%\documentclass[11pt,red,handout]{beamer}
%\documentclass[xcolor=pdftex,dvipsnames,table,11pt,red]{beamer}
\documentclass[table,11pt,red]{beamer}
% \documentclass[table,11pt,red,handout]{beamer}

%\usepackage[T1]{fontenc}
\usepackage{beamerthemesplit}
%\usepackage{wasysym}
%\usepackage{ulem}
\usepackage{amsmath}
\usepackage{amssymb}
\usepackage{arabtex}
\usepackage[francais]{babel}

\usepackage[cjkjis]{ucs} % UNICODE et, en cas d’ambiguité chinois/jp, du japonais
%\usepackage{arabtex}
\usepackage[utf8x]{inputenc} % Saisie en UTF8
\usepackage[C42,T1]{fontenc} % 2 encodages de police, C42, pour le japonais, et T1, pour le romain
\DeclareFontSubstitution{C42}{min}{m}{n}
% \usepackage[latin1]{inputenc}
% \usepackage[latin2]{inputenc}
%\usepackage[latin9]{inputenc}
\usepackage{fancyhdr}
\usepackage{graphicx}
%\usepackage{color}
\usepackage{xcolor}
%\usepackage{textcomp}
\usepackage{bera}
%\usepackage{times}
\usepackage{tabularx}
% \usepackage{multicol}
\usepackage{graphics}
\usepackage{pgf}
\usepackage{array}
% \usepackage{pifont}
% \usepackage{hyperref}
%\usepackage{lucidaso}
\definecolor{linacolor}{rgb}{148,0,0}
%%% macro : pour avoir des accents quelque soit le codage des caract?es du syst?e
%%
% \catcode`\?\active\def ?\'{e}}\catcode`\?\active\def ?\`{e}}
% \catcode`\?\active\def ?\`{a}}\catcode`\?\active\def ?\c{c}}
% \catcode`\=\active\def {\`{u}}\catcode`\?\active\def ?{\ae}}
% \catcode`\?\active\def ?\^{\i}}\catcode`\?\active\def ?\"{\i}}
% \catcode`\=\active\def {\^{u}}\catcode`\=\active\def {\"{u}}
% \catcode`\?\active\def ?\^{o}}\catcode`\?\active\def ?\^{e}}
% \catcode`\?\active\def ?\"{o}}\catcode`\?\active\def ?\^{a}}
% % \catcode`\?\active\def ?\"{O}}\catcode`\?\active\def ?\`{A}}
% \catcode`\=\active\def {{\oe}}



%\setbeamertemplate{footline}[frame number]


\mode<presentation>
{
  \usetheme{Warsaw}
  %\useoutertheme{infolines}
  \useoutertheme{shadow}
  \usecolortheme{lina}
}

%
%\usecolortheme{lina}
%\usecolortheme{whale}
\setbeamercolor*{frametitle}{parent=palette primary}
\setbeamerfont{block title}{size={}}


%\usefonttheme{boldserif}

 \setbeamertemplate{footline}{%
     \begin{beamercolorbox}[ht=2.5ex,dp=1.125ex,%
       leftskip=.3cm,rightskip=.3cm plus1fil]{title in head/foot}%
       \insertshorttitle\ (\insertshortauthor~-- \insertshortinstitute)%
       \hfill p. \insertframenumber/29%\inserttotalframenumber
     \end{beamercolorbox}%
   }




\setbeamertemplate{navigation symbols}{}
%\setbeamercovered{dynamic}

\logo{\includegraphics[width=0.1\textwidth]{./graphics/lina_small.png}}





%  \setbeamertemplate{footline}{%
%     \begin{beamercolorbox}[ht=2.5ex,dp=1.125ex,%
%       leftskip=.3cm,rightskip=.3cm plus1fil]{title in head/foot}%
%       \insertshorttitle\ (\insertshortauthor)%
%       \hfill\insertpagenumber
%     \end{beamercolorbox}%
%   }

% interligne de 1,5
% \renewcommand{\baselinestretch}{1.5}


%%% marges
%%
% \setlength{\hoffset}{-1cm}
% \setlength{\voffset}{-2cm}
% \setlength{\oddsidemargin}{2.4cm}
% \setlength{\evensidemargin}{1.2cm}
% \setlength{\topmargin}{0cm}
% \setlength{\headheight}{1.5cm}
% \setlength{\headsep}{0.5cm}
% \setlength{\textheight}{24cm}
% \setlength{\textwidth}{16.5cm}
% \setlength{\marginparsep}{0.5cm}
% \setlength{\marginparwidth}{0.5cm}
% \setlength{\footskip}{1.6cm}
%% saut entre les paragraphes
\setlength{\parskip}{.5\baselineskip}

\setlength\belowcaptionskip{1pt}
\setlength\abovecaptionskip{1pt}


%% profondeur de la table des maties
\setcounter{tocdepth}{1}     % Dans la table des matieres
% \setcounter{secnumdepth}{3}  % Avec un numero.

\title[Points d'ancrage dans les corpus comparables]{Influence des points d'ancrage pour l'extraction de lexique bilingue à partir de corpus comparables spécialisés}
%\subtitle{Extraction de lexique bilingue}
\author[Prochasson, Morin]{Emmanuel Prochasson\\Emmanuel Morin}

\institute[LINA]{Laboratoire d'Informatique de Nantes Atlantique }
\date{TALN 2009}

\begin{document}

% slide 1 
\frame {
	\titlepage
}
% 
% % slide 2

% slide 3

\frame{
  \frametitle{Introduction}
	\begin{itemize}
	\item Objectif : extraire un vocabulaire commun et l'aligner automatiquement pour constituer un lexique bilingue
	\item<2-> à partir de documents (multilingues) n'étant pas en relation de traduction
	\item<3-> $\rightarrow$ corpus \emph{comparables}
	\item<4-> «~\emph{Deux corpus de deux langues $l_1$ et $l_2$ sont dits comparables s'il existe une sous-partie non négligeable du vocabulaire du corpus de langue $l_1$, respectivement $l_2$, dont la traduction se trouve dans le corpus de langue $l_2$, respectivement $l_1$}~» (Déjean \& Gaussier, 2002)
	\end{itemize}
}

\frame {
	\frametitle{Plan}
	\tableofcontents
}



\section{Extraction de lexique bilingue}

\subsection{Corpus parallèles}
\frame{
	\frametitle{Corpus parallèles}
\begin{columns}
\begin{column}[l]{0.4\textwidth}
	\begin{itemize}
	\item<1-> Alignement basé sur la position et la distribution des mots dans les documents
	\item<2-> Alignement basé sur des mots déjà connus, utilisés comme points d'ancrage pour aligner leurs voisins
	\end{itemize}
	\visible<2->{(Véronis, 2000)}
\end{column}
\begin{column}[r]{0.6\textwidth}
\only<1-2>{\includegraphics[width=\textwidth]{./graphics/couloir-recherche.pdf}}
\only<3->{\includegraphics[width=\textwidth]{./graphics/couloir-recherche2.pdf}}
\end{column}
\end{columns}
}




\subsection{Corpus comparables}
\frame{
	\frametitle{Corpus comparables}
	\begin{itemize}
	\item Rapp (1995) et Fung (1995) introduisent l'alignement à partir de corpus \emph{non-parallèles}
	\item Tous deux s'appuient sur l'idée de caractériser le \emph{contexte} des mots à traduire, plutôt que des informations sur leurs positions
	\item<2-> Fung (1995) s'appuie sur les bigrammes (hétérogénéité à gauche/à droite), Rapp (1995) s'appuie sur les voisins rencontrés dans une fenêtre de taille fixe autour du mot à traduire.
	\end{itemize}
	\begin{block}{Firth, 1957}<3->
	\center
	«~On reconnaît un mot à ses fréquentations ~»
	\end{block}
}

\subsection{Approche directe}
\frame{
	\frametitle{Approche par traduction directe}
  \begin{columns}
  \begin{column}[l]{0.4\textwidth}
	\begin{itemize}
	\item<1-> \textbf{Construction de \emph{vecteurs de contexte}}
	\item<2-> Traduction des vecteurs sources vers la langue cible
	\item<3-> Calcul de la similarité entre vecteurs
	\item<6->$\rightarrow$ Liste ordonnée de candidats à la traduction
	\item<6->~~
  \end{itemize}
  \end{column}
  \begin{column}[r]{0.6\textwidth}
  \only<1>{\center\includegraphics[width=\textwidth]{./graphics/context-diabete-1.pdf}\\}
  \only<2>{\center\includegraphics[width=\textwidth]{./graphics/context-diabete-2.pdf}\\}
  \only<3>{\center\includegraphics[width=\textwidth]{./graphics/context-diabete-3.pdf}\\}
  \only<4>{\center\includegraphics[width=\textwidth]{./graphics/context-diabete-4.pdf}\\}
  \only<5>{\center\includegraphics[width=\textwidth]{./graphics/context-diabete-5.pdf}\\}
  \only<6->{\center\includegraphics[width=\textwidth]{./graphics/context-diabete-6.pdf}\\}
  \end{column}
  \end{columns}
}


  

\frame{
	\frametitle{Emphase : construction des vecteurs de contexte}
	\begin{itemize}
	\item Collecte de tous les élements (pertinents) dans une fenêtre donnée autour du mot à caractériser
	\item Calcul de l'\emph{association} (indépendance statistique) entre la tête du vecteur et ses élements
	\item<2-> Exemple : l'\emph{Information Mutuelle}
	\item<3->{$IM = log \frac{O}{E}$}
	\end{itemize}
	\begin{itemize}
	\item<4-> $\rightarrow$ Obtention d'un \emph{Motif d'Association}, pour un mot et ses voisins.
	\end{itemize}
}


\begin{frame}[plain]
\center
\includegraphics[width=\textwidth]{./graphics/pattern.pdf}\\
{\small \centerline{Motif d'association}}
\end{frame}


\begin{frame}[plain]
\center
\includegraphics[width=\textwidth]{./graphics/pattern-compare.pdf}\\

\end{frame}


\begin{frame}[plain]
\center
\includegraphics[width=\textwidth]{./graphics/pattern-compare2.pdf}\\
\end{frame}


\section{Détection et exploitation de points d'ancrage}
\frame{\tableofcontents[current]}
\subsection{Principes}

\frame{
    \frametitle{Problème}
	\begin{itemize}
	\item<+-> La construction de motifs significatifs nécessite des jeux de données volumineux
	\item<+-> Travail sur des textes spécialisés
	\item<+->\emph{Faibles volumes de matériaux textuels}
	\end{itemize}
}


\frame{
    \frametitle{Points d'ancrage}
    \begin{itemize}
    \item Idée : compenser le manque de données en s'appuyant sur des \emph{éléments de confiance}
    \item<2-> $\Rightarrow$ Points d'ancrage ! 
    \item<3-> Conceptuellement assez proche des points d'ancrage dans les corpus parallèles
	\begin{itemize}
	\item<4-> Exploiter les points d'ancrage pour rendre les vecteurs de contexte plus discriminants
	\item<4-> $\rightarrow$ rapprocher les vecteurs traductions
	\item<4-> $\rightarrow$ éloigner les vecteurs non traductions
	\end{itemize}
    \end{itemize}

	\begin{block}{Propriétés}<5->
    \begin{enumerate}
    \item<5-> Pertinents vis-à-vis des thématiques du corpus
    \item<5-> Détectables automatiquement
    \item<5-> Traductions stables
    \end{enumerate}
	\end{block}
}

\frame{
    \frametitle{Exploitation des points d'ancrage}
    \begin{itemize}
    \item Idée : construire le motif d'association en priorité sur les points d'ancrage, puis sur les autres éléments
    \item Augmentation articielle du score d'association des points d'ancrage
	\item<+-> «~Déformation~» du motif d'association en faveur des points d'ancrage
    \end{itemize}
}

\begin{frame}[plain]
\center
\includegraphics[width=\textwidth]{./graphics/pattern-vierge.pdf}\\
\end{frame}

\begin{frame}[plain]
\center
\includegraphics[width=\textwidth]{./graphics/pattern-highlight.pdf}\\
\end{frame}

\begin{frame}[plain]
\center
\includegraphics[width=\textwidth]{./graphics/pattern-shifted.pdf}\\
\end{frame}

\begin{frame}[plain]
\center
\includegraphics[width=\textwidth]{./graphics/pattern-compare-shifted.pdf}\\
\end{frame}

\begin{frame}[plain]
\center
\includegraphics[width=\textwidth]{./graphics/pattern-compare-shifted2.pdf}\\
\end{frame}

\begin{frame}[plain]
\center
\includegraphics[width=\textwidth]{./graphics/pattern-compare-shifted3.pdf}\\
\end{frame}


\subsection{Exemple de points d'ancrage}

\frame{
	\frametitle{Contexte}
	Utilisation d'un corpus comparable anglais, japonais, français
	\begin{itemize}
	\item Domaine \emph{médical}
	\item Thème \emph{Alimentation et diabète}
	\item Registre \emph{scientifique}
	\item Environ 250\,000 mots par partie
	\end{itemize}
}

\frame{
    \frametitle{Points d'ancrage (2)}
    Deux types de points d'ancrage identifiés, respectant les propriétés :
    \begin{itemize}
    \item Translittérations japonaises (et leurs correspondances en français et en anglais)
    \item Composés savants anglais/français (et leurs traductions en japonais)
    \end{itemize}
}

\frame{
	\frametitle{Translittérations}
	 \begin{itemize}
	\item Adaptation phonétique d'un mot aux contraintes du japonais
	\item Exemple : インスリン/i-n-su-ri-n (insulin/insuline)
	\item Facile à identifier (syllabaire dédié)
	\item Alignement automatique sur la base de la prononciation
	\item Couvre un vocabulaire spécifique, dans le cas des documents \emph{scientifique} (Ito, 2007)
	\item Emprunt à l'anglais, mais alignement possible avec le français
	\item Détectées avec un outil dédié à l'alignement anglais/japonais (Tsuji, 2005)
	\end{itemize}
}

\frame{
	\frametitle{Composés savants}
	\begin{itemize}
	\item Mots construits sur des racines grecques et latines (Namer, 2005)
	\item<1->\emph{psychologie}, construit avec le préfixe \emph{psycho-} et le suffixe \emph{-logie}
	\item<2->Dérivations régulières entre l'anglais et le français (Claveau, 2007)
	\item<2->\emph{logy} (en) $\rightarrow$ \emph{logie} (fr)
	\item<3-> Caractéristique d'un vocabulaire \emph{scientifique}
	\item<4-> Détectés à l'aide d'une liste d'affixes
	\end{itemize}
}


\subsection{Expérience -- Résultats}

\frame{
	\frametitle{Protocole}
	Trois expériences
	\begin{enumerate}
	\item «~\emph{Témoin}~»
	\item Translittérations
	\item Composés savants
	\end{enumerate}
	\begin{itemize}
	\item Points d'ancrage \emph{translittérations} : 589 (en/jp) 526 (fr/jp)
	\item Points d'ancrage \emph{composés savants} : 604 (en/jp) 819 (fr/jp)
	\end{itemize}
	\begin{itemize}
	\item Utilisation d'une liste de référence de 98 termes
	\item Taille de fenêtre: 25
	\end{itemize}
}

\frame{
	\frametitle{Résultats}

\begin{table}[h!t!p]
\center
\footnotesize
% \begin{tabular}{l<{\onslide<1->}|c<\onslide<1->r@{,}<\onslide<1->c<{\onslide<3->}c<{\onslide<4->}c<{\onslide}c}
\begin{tabular}{l|c<{\onslide<2->}|c<{\onslide<3->}|c<{\onslide}}

 & \emph{Témoin} 	& \emph{Translittérations} & \emph{Composés Savants} \\
\hline
Anglais/Japonais ($Top_1$) 	& 17,1\,\%	& 20,2\,\% [+18,2\,\%]& 20,2\,\% [+18,2\,\%]\\
Anglais/Japonais ($Top_{10}$)	& 36,3\,\%	& 39,3\,\% [+~~8,2\,\%]	& 40,4\,\% [+11,2\,\%]\\
\hline
Français/Japonais ($Top_1$)		& 20,4\,\%	& 20,4\,\% [+~~0,0\,\%]& 22,4\,\% [+10,0\,\%]\\
Français/Japonais ($Top_{10}$)	& 36,7\,\%	& 37,8\,\% [+~~2,8\,\%] & 38,8\,\% [+~~5,6\,\%]\\
\end{tabular}
\caption{\label{resultat-1}Résultats de l'alignement anglais-japonais et français-japonais}
\end{table}
}

\section{Conclusions}
 

\frame{
  \frametitle{Analyse}
  \begin{itemize}
  \item Effet des points d'ancrage sur les résultats de l'alignement~:
  \begin{itemize}
  \item $\rightarrow$ léger reclassement des candidats bien classés ($Top < 15$)
  \item $\rightarrow$ large reclassement des candidats mal classés ($Top > 50$)
  \end{itemize}
  \end{itemize}
  \begin{itemize}
  \item Amélioration globale et significative des résultats \visible<2->{$\rightarrow$ même si améliorations $Top_1$ et $Top_{10}$ faibles}
  \end{itemize}
}



\frame{
  \frametitle{Conclusion, discussion}
  \begin{itemize}
  \item Nouvelle hypothèse pour l'alignement de lexique à partir de corpus comparables spécialisés
  \item Résultats globalement significatifs
  \item Hypothèse extensible à d'autres types de vocabulaire
  \end{itemize}
  \begin{itemize}
  \item<2-> Expérience à reproduire avec d'autres points d'ancrage
  \item<2-> À définir en fonction des couples de langues impliquées
  \item<2-> Utilisation d'autres techniques transversales en TALN
  \item<2-> (détection des cognats, techniques de RI\ldots)
  \end{itemize}
}

\frame{
  \frametitle{Fin}
  \center
  Merci de votre attention
}

\frame{
  \frametitle{Prise en compte des points d'ancrage}
  \begin{equation}
  {assoc_{(PA)}}_a^v = assoc_a^v + \beta
  \end{equation}

}

\frame{
  \frametitle{Influence des points d'ancrage}
  \begin{columns}
  \begin{column}[l]{0,5\textwidth}
  \includegraphics[width=\textwidth]{./graphics/graphe-b.pdf}
  \end{column}
  \begin{column}[r]{0,5\textwidth}
  \includegraphics[width=\textwidth]{./graphics/graphe-c.pdf}
  \end{column}
  \end{columns}
}



\end{document}
